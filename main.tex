% Datei: Documents/latex/conconi_ergometer/main.tex
% Begonnen: 12.11.2014 
% Letzte Änderung: 12.11.2014 
%
% Vorlage für einen umfangreichen Artikel mit Titelei, Abstract,
% Inhalts- und Literaturverzeichnis

% Einteilung des Paperformates, des Satzspiegels und eine Bindekorrektur von 1 cm
\documentclass[a4paper,DIV11,BCOR1cm]{scrartcl}

\usepackage[english,ngerman]{babel}
\usepackage[T1]{fontenc}
\usepackage[utf8]{inputenc}
\usepackage{lmodern}	% Schrift "Latin Modern" laden
\usepackage{mdwlist}	% Für enger gesetzte Listen

\usepackage{typearea}	% Einstellung des Satzspiegels, siehe Latex Hacks, Hack #31
\usepackage{booktabs} 	% schönere Tabellen, siehe WAsmL, p. 124

\usepackage{babelbib}
\usepackage{url}		% wird von babelbib gebraucht

% Die Option dvipdfmx und das Paket bmpsize verhindern
% die Fehlermeldung von latex:
% "cannot determine size of graphic"
\usepackage[dvipdfmx]{graphicx}
\usepackage{bmpsize}

\usepackage{apacite}

\usepackage{fancyhdr}
\pagestyle{fancy}

\setcounter{secnumdepth}{2}

\begin{document}

% \titlehead{Conconi-Test}
\lhead{Conconi-Test auf Ergometer}

% \subject{Conconi-Test}
\title{Durchführen des Conconi-Tests mit einem Ergometer}
\author{Dr. med. Marco Strehlerh}
%\publishers{}
\date{12.\,11.\,2014}

\maketitle
%\clearpage

\begin{abstract}
Der Conconi-Test ist ein Leistungstest.
Er kann in verschiedenen Leistungszentren durchgeführt werden.
Dieser Text stellt ein Protokoll vor um den Conconi-Test einfach und unkompliziert
selber durchzuführen.
\end{abstract}

\tableofcontents

\section{Grundlagen}

\subsection{Der Conconi-Test}

\subsection{Vergleich verschiedene Protokolle}

Francesco Conconi entwickelte \ldots


\section{Vereinfachtes Protokoll für Ergometer}

\subsection{Vorbereitung}

Am Vortag keine intensive oder lange Trainings und Wettkämpfe.
Rund 2 Stunden vor dem Test sollte keine grössere Mahlzeite eingenommen werden.

Unmittelbar vor Test: Brustgurt Polar. Verbindung mit Cybex kontrollieren.

\subsection{Aufwärmen}
10 Minuten Einfahren auf der Belastungsstufe von 50-100 Watt aufwärmen. Da der Test mit 85 RPM gefahren wird, soll gleich diese Frequenz gewählt werden. 

\subsection{Durchführen}

Einstellung: Manuell, Zeit 60 Minuten, Stufe 8.

Auf dem Ergometer jede Minute eine Belastungsstufe höher schalten, bei einer Trittfrequenz von 85 U/min.
Die gemessenen Watt-Zahlen für eine bestimmte Trittfrequenz finden sich in Google-Drive (File "Workout/Conconi-Test")

Für den Test wird mit Stufe 8 begonnen u. jede Minute um eine Stufe erhöht. Die Herzfrequenz am Ende jeder Minute, vor dem Hochschalter der Stufe wird notiert.

\subsection{Auswertung}

Die gemessenen Wertepaare aus Herzfrequenz und Watt werden in ein Diagramm eingetragen und ausgewertet.
Laut Conconi ist die anaerobe Schwelle an dem Punkt erreicht (Deflexionspunkt), an dem die lineare Beziehung zwischen Herzfrequenz und der Geschwindigkeit in eine flachere Kurve übergeht, die Kurve also – bildlich gesprochen – einen Knick nimmt.

\cite{*}

\bibliographystyle{apacite}
\bibliography{conconi-literatur.bib}

\end{document}


